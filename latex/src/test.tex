
\documentclass[11pt, letterpaper]{article}
\usepackage[utf8]{inputenc}
\usepackage{graphicx}
\usepackage{xcolor}
\usepackage{enumitem}
\usepackage{listings}


\definecolor{dkgreen}{rgb}{0,0.6,0}
\definecolor{gray}{rgb}{0.5,0.5,0.5}
\definecolor{mauve}{rgb}{0.58,0,0.82}

\lstset{
  language=C++,
  aboveskip=3mm,
  belowskip=3mm,
  showstringspaces=false,
  columns=flexible,
  basicstyle={\small\ttfamily},
  numbers=left,
  numberstyle=\small\color{blue},
  keywordstyle=\color{blue},
  commentstyle=\color{dkgreen},
  stringstyle=\color{mauve},
  breaklines=true,
  breakatwhitespace=true,
  tabsize=4
}

\author{Jörg Barkoczi}
\title{\vspace{-2.0cm}Chapter 2 Problem Solutions}
\date{ }

\begin{document}
\maketitle
\section*{2-2 \textit{Correctness of bubblesort}}
\begin{enumerate}[label=\alph*)]
    \item
        What we need to prove:
        \begin{itemize}
            \item
                valid loop invariant
                \begin{itemize}
                    \item Initialization
                    \item Maintenance
                    \item Termination
                \end{itemize}
            \item
                that it contains all original elements.
        \end{itemize}
    \item
        \textbf{Loop invariant (2-4)}:

        The for-loop in lines 2-4 will maintain a loop invariant
        such that the element at position $A[j]$ will be the smallest
        in the subarray $A[n-j]$.

        \textbf{Initialization}:: $A[n-j]$ will contain a single element, thus it'll be the smallest.

        \textbf{Maintenance}: If $A[j-1] > A[j]$, both will be swapped, thus maintaining the
        loop invariant.
       
        \textbf{Termination}: Element $A[j]$ will be the smallest in the subsequence $A[n-j]$.

   \item
       \textbf{Loop invariant (1-4)}:

        Each iteration sorts the sequence, such that the $i$th smallest element will 
        be on the $i$th position. This will partition the sequence in the following 
        subsequences after each iteration:\\~\\
        $A[1...i]$ which is sorted such that $A[1] \leq A[2] \leq \dots \leq A[i]$.\\
        $A[i+1 \dots n]$ which is unsorted.

        In the end the $n-1$ smallest elements will be positioned at positions 
        $A[1 \dots n-1]$ respectively, thus leaving the $n$th smallest element - which
        is the biggest - at position $n$.

        \textbf{Initialization}: $i = 1$, therefore $A[1\dots i]$ contains only a
        single element, which is trivially sorted.

        \textbf{Maintenance}: $i$ is incremented, thus fullfilling the loop invariant 
        yielding a sorted subarray $A[1\dots i]$.

        \textbf{Termination}: When $i = n-1$, the sorted subsequence will be of size
        $n - 1$ and will contain the $n-1$th smallest elements, thus leaving the last 
        element $A[n]$ which must be the largest.

    \item
        The worst case running time occurs everytime, because regardless of how sorted
        the sequence already is, the inner loop will always iterate $n-i+1$ times, thus
        giving a total runtime of:
        \begin{equation}
            \Theta ((n-1)(n-i+1)) \Leftrightarrow \Theta(n^2-ni+i-1) \equiv \Theta(n^2)
        \end{equation}
\end{enumerate}
\pagebreak
\section*{2-3 - \textit{Correctness of Horner's rule}}
\begin{enumerate}[label=\alph*)]
    \item
        The running time of the given codefragment is $\Theta (n)$.
    \item
        ~\\
        \begin{lstlisting}
        template<typename X>
        long double naive_polynomial_evaluator(const X x, const std::vector<int>& arr2)
        {
            std::vector<int> powers(arr2.size());

            powers[0] = 1;
            powers[1] = x;
            for (int i = 2; i < powers.size(); ++i)
            {
                powers[i] = x * powers[i-1];
            }
            long double y = 0;

            for (int i = 0; i < arr2.size(); ++i)
            {
                y += arr2[i] * std::pow(x, arr2.size() - 1 - i);
            }

            return y;
        }
        \end{lstlisting}
        The running time of this algorithm is $\Theta (n^2)$ and thus clearly slower 
        than Horner's rule.
    \item
        Let $y'$ denote the value of $y$ after each iteration.\\
        Given that $y' = a_i + xy$ and $y = \sum_{k=0}^{n-(i+1)} a_{i+k+1}x^k$ we 
        can rewrite it as \\$y'=a_i + x \sum_{k=0}^{n-(i+1)} a_{i+k+1}x^k\\
        \Leftrightarrow
        a_i + \sum_{k=0}^{n-(i+1)} a_{i+k+1}x^{k+1}\\
        \Leftrightarrow
        a_i x_0 + \sum_{k=0}^{n-(i+1)} a_{i+k+1}x^{k+1}\\
        \Leftrightarrow
        \sum_{k=-1}^{n-(i+1)} a_{i+k+1}x^{k+1}\\
        \Leftrightarrow
        \sum_{k=0}^{n-i} a_{i+k}x^{k}\\
        $
        Now at termination where $i = 0$, we get
        \begin{equation}
        \sum_{k=0}^{n-0} a_{0+k}x^{k} \Leftrightarrow \sum_{k=0}^{n} a_{k}x^{k}
        \end{equation}
        which is indeed the same as the given summation.
    \item
        As we have just proved, we can say that the given code fragment correctly
        evaluates a polynomial characterized by the coefficients $a_0,a_1,\dots,a_n$.
\end{enumerate}
\pagebreak
\section*{2-4 - \textit{Inversions}}
\begin{enumerate}[label=\alph*)]
    \item
    The five inversions of the array ${2,3,8,6,1}$ are as follows:
        \begin{enumerate}
            \item (8,6)
            \item (2,1)
            \item (3,1)
            \item (8,1)
            \item (6,1)
        \end{enumerate}
    \item
        An array from the set ${1,2,\dots,n}$ has the most inversions, if it has the 
        property $A[n]>A[n-1]>\dots>A[1]$. It will have $\frac{n(n-1)}{2}$ inversions.
    \item
        Each inversion corresponds to an iteration of the inner loop in the insertion sort
        algorithm.\\
        Given an array $A[1\dots n]$ and a function $I(x)$ which takes an element of A
        and outputs the number of inversion for that element, we in total have 
        $\sum_{i=1}^{n} I(A[i])$ iterations of the inner loop.
     \item
         ~\\
        Given the hint of modifying our existing merge sort algorithm, we can 
        add another statement of constant time on line 33, which calculates the remaining
        elements in array $v1$ - which is left of $v2$ - and thus reflects the number
        of inversions for element $v2[j]$. We also change the return value to the number 
        of inversions, such that it adds up and propagates through the recursion. Since we 
        only added operations of constant time, the worst case running time of the algorithm stays at
        $\Theta (nlgn)$ and thus satisfies the required worst case running time.
        \begin{lstlisting}
        template<typename T>
        int merge(std::vector<T>& v, int l, int r, int p)
        {
            int s1 = p - l + 2;
            int s2 = r - p + 1;

            std::vector<T> v1(s1);
            for (int i = 0; i < s1-1; ++i)
            {
                v1[i] = v[l+i];
            }
            v1[s1-1] = std::numeric_limits<T>::max();

            std::vector<T> v2(s2);
            for (int i = 0; i < s2-1; ++i)
            {
                v2[i] = v[p+i+1];
            }
            v2[s2-1] = std::numeric_limits<T>::max();
            
            int inv = 0;
            int i = 0;
            int j = 0;
            for (int a = l;a <= r; ++a)
            {
                if (v1[i] <= v2[j])
                {
                    v[a] = v1[i++];
                }
                else
                {
                    v[a] = v2[j++];
                    inv += s1 - i - 1;
                }
            }
            return inv;
        }

        template<typename T>
        int merge_sort(std::vector<T>& v, int l, int r)
        {
            auto inv = 0;
            if (l < r)
            {
                int p = (l + r) / 2;
                inv += merge_sort(v, l, p);
                inv += merge_sort(v, p+1, r);
                inv += merge<T>(v, l, r, p);
            }
            return inv;
        }
        \end{lstlisting}
\end{enumerate} 
\end{document}
